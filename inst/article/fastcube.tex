\documentclass[12pt,english]{article}\usepackage[]{graphicx}\usepackage{xcolor}
% maxwidth is the original width if it is less than linewidth
% otherwise use linewidth (to make sure the graphics do not exceed the margin)
\makeatletter
\def\maxwidth{ %
  \ifdim\Gin@nat@width>\linewidth
    \linewidth
  \else
    \Gin@nat@width
  \fi
}
\makeatother

\definecolor{fgcolor}{rgb}{0.251, 0.251, 0.282}
\newcommand{\hlnum}[1]{\textcolor[rgb]{0.125,0.125,1}{#1}}%
\newcommand{\hlstr}[1]{\textcolor[rgb]{0.125,0.125,1}{#1}}%
\newcommand{\hlcom}[1]{\textcolor[rgb]{1,0,0.753}{\textit{#1}}}%
\newcommand{\hlopt}[1]{\textcolor[rgb]{0.251,0.251,0.282}{#1}}%
\newcommand{\hlstd}[1]{\textcolor[rgb]{0.251,0.251,0.282}{#1}}%
\newcommand{\hlkwa}[1]{\textcolor[rgb]{0,0.533,0.345}{\textbf{#1}}}%
\newcommand{\hlkwb}[1]{\textcolor[rgb]{0.439,0.251,1}{\textbf{#1}}}%
\newcommand{\hlkwc}[1]{\textcolor[rgb]{0.529,0,0.184}{\textbf{#1}}}%
\newcommand{\hlkwd}[1]{\textcolor[rgb]{0.251,0.251,0.282}{\textbf{#1}}}%
\let\hlipl\hlkwb

\usepackage{framed}
\makeatletter
\newenvironment{kframe}{%
 \def\at@end@of@kframe{}%
 \ifinner\ifhmode%
  \def\at@end@of@kframe{\end{minipage}}%
  \begin{minipage}{\columnwidth}%
 \fi\fi%
 \def\FrameCommand##1{\hskip\@totalleftmargin \hskip-\fboxsep
 \colorbox{shadecolor}{##1}\hskip-\fboxsep
     % There is no \\@totalrightmargin, so:
     \hskip-\linewidth \hskip-\@totalleftmargin \hskip\columnwidth}%
 \MakeFramed {\advance\hsize-\width
   \@totalleftmargin\z@ \linewidth\hsize
   \@setminipage}}%
 {\par\unskip\endMakeFramed%
 \at@end@of@kframe}
\makeatother

\definecolor{shadecolor}{rgb}{.97, .97, .97}
\definecolor{messagecolor}{rgb}{0, 0, 0}
\definecolor{warningcolor}{rgb}{1, 0, 1}
\definecolor{errorcolor}{rgb}{1, 0, 0}
\newenvironment{knitrout}{}{} % an empty environment to be redefined in TeX

\usepackage{alltt}
\usepackage[a4paper]{geometry}
% \usepackage{fullpage}
\usepackage{natbib}
\usepackage{algorithm}
\usepackage{graphicx}

% \usepackage{mathptmx}
\usepackage{alphalph}

\usepackage[utf8]{inputenc}
\usepackage[T1]{fontenc}
\usepackage{tikz}
\usepackage{float}

\usepackage[colorlinks = true,linkcolor = blue, citecolor = blue]{hyperref}
\usepackage{booktabs}
\usepackage{amssymb,amsmath}%,natbib,rotating,mfpic} %amsmath}


% \usepackage{layouts}

\def\aaa{a}
\def\aaag{\textbf{a}}
\def\R{\mathbb{R}}
\def\N{\mathbb{N}}
\def\tb{{\bf t}}
\def\zerog{{\bf 0}}
\def\trace{{\rm trace}}
\def\rang{{\rm rang}}
\def\Gammag{\mathbf{\Gamma}}
\def\Deltag{\mathbf{\Delta}}
\def\Thetag{\mathbf{\Theta}}
\def\Lambdag{\mathbf{\Lambda}}
\def\Xig{\mathbf{\Xi}}
\def\Pig{\mathbf{\Pi}}
\def\Sigmag{\mathbf{\Sigma}}
\def\Pig{\mathbf{\Pib}}
\def\Phig{\mathbf{\Phi}} %
\def\Psig{\mathbf{\Psi}} %}}
\def\Omegag{\mathbf{\Omega}} %}}
\def\Upsilong{\mathbf{\Upsilon}} %}}


\def\Var{{\rm var}}
\def\Min{{\rm Min}}
\def\Cov{{\rm cov}}
\def\Corr{{\rm corr}}
\def\Im{{\rm Im}}
\def\Ker{{\rm Ker}}
\def\SC{{\rm SC}}
\def\CM{{\rm CM}}
\def\diag{{\rm diag}}
\def\E{{\rm E}}
\def\Pr{{\rm Pr}}
\def\Vb{{\bf V}}




%%%%%%%%%%%%%%%%%%%%%%%%%%%%%%%%%%%%%%%%%%%%%%%%%%%%%%
\def\alphag{\boldsymbol{\alpha}} %$\alpha$
\def\betag{\boldsymbol{\beta}} %$\beta$
\def\gammag{\boldsymbol{\gamma}} %$\gamma$
\def\deltag{\boldsymbol{\delta}} %$\delta$
\def\epsilong{\boldsymbol{\epsilon}} %$\epsilon$
\def\varepsilong{\boldsymbol{\varepsilon}} %$\varepsilon$
\def\etag{\boldsymbol{\eta}} %$\eta$
\def\thetag{\boldsymbol{\theta}} %$\theta$
\def\iotag{\boldsymbol{\iota}} %$\iota$
\def\kappag{\boldsymbol{\kappa}} %$\kappa$
\def\lambdag{\boldsymbol{\lambda}} %$\lambda$
\def\mug{\boldsymbol{\mu}} %$\nu$
\def\nug{\boldsymbol{\nu}} %$\mu$
\def\xig{\boldsymbol{\xi}} %$\xi$
\def\pig{\boldsymbol{\pi}} %$\pi$
\def\rhog{\boldsymbol{\rho}} %$\rho$
\def\sigmag{\boldsymbol{\sigma}} %$\sigma$
\def\taug{\boldsymbol{\tau}} %$\tau$
\def\upsilong{\boldsymbol{\upsilon}} %$\upsilon$
\def\phig{\boldsymbol{\phi}} %$\phi$
\def\chig{\boldsymbol{\chi}} %$\chi$
\def\psig{\boldsymbol{\psi}} %$\psi$
\def\omegag{\boldsymbol{\omega}} %$\omega$

%%%%%%%%%%%%%%%%%%%%%%%%%%%%%%%%%%%%%%%%%%%%%%%%%%%%%%

\def\tb{{\bf t}}
\def\zerog{{\bf 0}}
\def\trace{{\rm trace}}
\def\rang{{\rm rang}}
\def\Gammag{\mathbf{\Gamma}}
\def\Deltag{\mathbf{\Delta}}
%\def\Thetag{\mathbf{\Theta}}
\def\Lambdag{\mathbf{\Lambda}}
%\def\Xig{\mathbf{\Xi}}
%\def\Pig{\mathbf{\Pi}}
\def\Sigmag{\mathbf{\Sigma}}
%\def\Pig{\mathbf{\Pib}}
%\def\Phig{\mathbf{\Phi}} %
%\def\Psig{\mathbf{\Psi}} %}}
\def\Omegag{\mathbf{\Omega}} %}}


\def\Xbu{{\bf X}_1}
\def\Xbd{{\bf X}_2}
\def\Xbt{{\bf X}_3}
%\def\Yb{{\bf Y}}
\def\Db{{\bf D}}
\def\Eb{{\bf E}}
\def\Fb{{\bf F}}
\def\Tb{{\bf T}}

\def\nb{{\bf n}}
\def\ab{{\bf a}}
\def\sb{{\bf s}}
\def\eb{{\bf e}}
\def\gb{{\bf g}}
\def\rb{{\bf r}}
\def\cg{{\bf c}}

\def\yb{{\bf y}}
\def\zb{{\bf z}}
\def\bb{{\bf b}}
\def\cb{{\bf c}}
\def\fb{{\bf f}}
\def\ub{{\bf u}}
\def\xb{{\bf x}}
\def\vb{{\bf v}}
\def\wb{{\bf w}}
\def\Ab{{\bf A}}
\def\Bb{{\bf B}}
\def\Cb{{\bf C}}
\def\Mb{{\bf M}}
\def\Nb{{\bf N}}
\def\Cg{{\bf C}}
\def\Ib{{\bf I}}
\def\Xb{{\bf X}}
\def\Zb{{\bf Z}}
\def\Pb{{\bf P}}
\def\Db{{\bf D}}
\def\Rb{{\bf R}}
\def\Qb{{\bf Q}}
\def\Sb{{\bf S}}
\def\Ub{{\bf U}}
\def\Wb{{\bf W}}
\def\0b{{\bf 0}}
\def\1b{{\bf 1}}

\newtheorem{exmpl}{Example}[section]
\newtheorem{defn}{Definition}

\newcommand{\myalphafoot}
{
\renewcommand{\thefootnote}{\alph{footnote}}
}

\title{fast cube implementation}
\myalphafoot
\author{\myalphafoot Rapha\"el Jauslin\footnotemark[1]~, Esther Eustache\footnotemark[1]~ and Yves Till\'e\footnotemark[1]}
\date{}
\footnotetext[1]{Institute of statistics, University of Neuchatel, Av. de Bellevaux 51, 2000 Neuchatel, Switzerland\\ (E-mail: raphael.jauslin@unine.ch)}

\IfFileExists{upquote.sty}{\usepackage{upquote}}{}
\begin{document}

\maketitle




\begin{abstract}


\textbf{Key words}: optimal design, spread sampling, stratification
\end{abstract}
\newpage

%-----------------------------------------------------------------------------------
% Introduction
%-----------------------------------------------------------------------------------

\section{Introduction}

%-----------------------------------------------------------------------------------
% Notation
%-----------------------------------------------------------------------------------

\section{Notation}


Consider a finite population $U$ of size $N$ whose units can be defined by labels $k\in\{1,2,\dots,N\}$. Let $\mathcal{S} = \{s | s\subset U\}$ be the set of all possible samples. A sampling design is defined by a probability distribution $p(.)$ on $\mathcal{S}$ such that

$$
p(s) \geq 0 \text{ for all } s\in \mathcal{S} \text{ and }\sum_{ s\in \mathcal{S}}p(s) = 1.
$$

A random sample $S$ is a random vector that maps elements of $\mathcal{S}$ to an $N$ vector of 0 or 1 such that $\textrm{P}(S = s) =
p(s)$. Define $a_k(S)$, for $k = 1,\dots,N$:

$$
\aaa_k =
\left\{\begin{array}{lll} 1 & \text{ if } k\in S\\ 0 & \text{ otherwise} . \end{array} \right.
$$

Then a sample can be denoted by means of a vector notation:
$
 \aaag^\top = (\aaa_1,\aaa_2,\dots,\aaa_N).
$ For each unit of the population, the inclusion probability $0\leq\pi_k\leq 1$ is defined as the probability that unit $k$ is selected into sample $S$:
\begin{equation*}\label{eq:pik}
 \pi_k = \textrm{P}(k \in S) = \textrm{E}(\aaa_k) =  \sum_{s\in S | k \in s} p(s), \text{ for all } k\in U.
\end{equation*}

Let $\pig^\top=(\pi_1,\dots,\pi_N)$ be the vector of inclusion probabilities. Then, $\textrm{E}({\aaag})=\pig.$ Let also $\pi_{k\ell}$ be the probability of selecting the units $k$ and $\ell$ together in the sample, with $\pi_{kk} = \pi_k$. The matrix of second-order inclusion probabilities is given by $\Pi = \E(\aaag\aaag^\top)$. In many applications, inclusion probabilities are such that samples have a fixed size $n$. Let the set of all samples that have fixed size equal to $n$ be defined by
 \begin{equation*}\label{eq:sn} \mathcal{S}_n = \left\{ \aaag\in \{0,1\}^N ~~\bigg|~~ \sum_{k
= 1}^N \aaa_k = n \right\} .
 \end{equation*}

The sample is generally selected with the aim of estimating some population parameters. Let $y_k$ denote a real number associated with unit $k\in U$, usually called the variable of interest. For example, the total
$$
Y=\sum_{k\in U} y_k
$$
can be estimated by using the classical Horvitz-Thompson estimator of the total defined by
\begin{equation}\label{eq:HT}
\widehat{Y}_{HT} = \sum_{k\in U} \frac{y_k a_k}{\pi_k}.
\end{equation}


Usually, some auxiliary information $\xb_k^\top = (x_{k1},x_{k2},\dots,x_{kq}) \in\mathbb{R}^q$  regarding the population units is available.
In the particular case of spatial sampling, a set of spatial coordinates  $\zb_k^\top = (z_{k1},z_{k2},\dots,z_{kp}) \in \mathbb{R}^p$ is supposed to be available, where $p$ is the dimension of the considered space. A sampling design is said to be balanced on the auxiliary variables $x_k$ if and only if it satisfies the balancing equations
\begin{equation*}\label{eq:balance}
  \widehat{\Xb} = \sum_{k\in S} \frac{\xb_k}{\pi_k} = \sum_{k\in U} \xb_k = \Xb.
\end{equation*}


%-----------------------------------------------------------------------------------
%	Balanced Sampling
%-----------------------------------------------------------------------------------
\section{Balanced Sampling}



%-----------------------------------------------------------------------------------
%	Cube Method
%-----------------------------------------------------------------------------------
\section{Cube Method}


%-----------------------------------------------------------------------------------
%	Reduction
%-----------------------------------------------------------------------------------
\section{Reduction}

%-----------------------------------------------------------------------------------
%	Simulation
%-----------------------------------------------------------------------------------

\section{Simulation}

%-----------------------------------------------------------------------------------
%	Discussion
%-----------------------------------------------------------------------------------

\section{Discussion}


\end{document}
